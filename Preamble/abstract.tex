This project aims to design and develop a custom, high-level language compiler. The compiler will be implemented incrementally, allowing it to accept a significant language subset and generate assembly code for a selected target architecture. The motivation behind this project stems from the need for a compiler that adheres to fundamental software engineering principles while offering concise and expressive concepts.

The development process will follow a step-by-step approach, from designing the language's grammar to adding more complex features. Each stage will yield a fully working compiler for a specific subset of the source language. By adopting this incremental methodology, the project seeks to provide a deep understanding of the different components of a compiler, including lexical analysis, parsing, semantic analysis, and code generation.

The resulting compiler will provide a practical resource for students, researchers, and developers interested in compiler design and contribute to advancing the field. It will demonstrate the application of various concepts and techniques in language implementation, software optimization, and system-level programming.

Additionally, the project acknowledges the importance of considering the target architecture during compilation. The generated assembly code will be tailored to the specific requirements and capabilities of the selected architecture, ensuring optimal performance and compatibility.

This project strives to deliver a well-designed and efficient compiler that bridges the gap between theoretical concepts and practical implementation. The project aims to contribute to language implementation and empower developers to build robust and efficient software systems by combining principles of software engineering, compiler design, and target architecture optimization.

\vspace{2em}
\emph{\textbf{Keywords: Compiler, Language Design, Programming Languages, Raspberry Pi}}